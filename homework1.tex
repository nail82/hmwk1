% This is the start of homework 1
% I think I will start from latex scratch and build it up
% as an article, just to cut down on some of the cruft.

\title{CS690 Homework 1}
\author{Ted Satcher}
\documentclass[12pt]{article}
\usepackage{cite}
\date{\today}
\begin{document}
\maketitle

\section*{Definitions}
\subsection*{Process switch}
A process switch is when the operating system suspends a process
currently executing, saves the state of the process switches to, and
begins executing another process in the ready queue.  This is a key
feature of a multiprogramming operating system.  It enables multiple
programs to execute on a single CPU ~\cite{stallings}.

\subsection*{Multiprogramming}
Multiprogramming is the feature of an operating system that permits
multiple programs to execute concurrently on a CPU. Generally, a
number of executing programs are kept in main memeory and the
operating system switcheds between the programs based on various
criteria ~\cite{stallings}.

\subsection*{System call}
A system call is a request mad by a user program to the operating
system.  The user request is for data or a resource that requires the
execution of priviledged instructions.  System calls are the means by
which programs can do work requiring priviledged access by proxying
those requests through the operating system ~\cite{stallings}.

\subsection*{Semaphore}
A semaphore is a synchronization primitive, invented by Dijkstra, that
can be used by cooperating processes or threads to access a shared
resource or data.  Semaphores are implemented as a simple counter that
can only be incremented or decremented using special, atomic function
calls.  these calls may or man not block based on the value of the
semaphore.  Semaphores are one of several methodes processes and
threads can use to coordinate their activities ~\cite{stallings}.
Semaphores can be used to implement other types of synchronization
primitives such as mutexes.

\section*{Process Switch Triggers}
List two events that can trigger a process switch.

\section*{Live Migration}
Explain what ``live'' migration meand and cite some of its benefits

\section*{Design Principles of a Multikernel}
State and explain three design principles for a multikern

\section*{Scalability of OS Locks}

\bibliography{homework1}{}
\bibliographystyle{plain}
\end{document}